\addcontentsline{toc}{chapter}{Abstract}

\begin{abstract}
\Large\centering{Wireless brain-machine interface system based on multi-channel visual flickers}

\normalsize
\begin{flushright}
Karunyapas  Dangruan  55090005\\
Pongrawee  Jutadhammakorn 55090033\\
Siwatch  Luxsameepicheat  55090049\\
Academic Year 2015
\end{flushright}

\justify
\hspace{1.5cm} Brain-Machine Interface (BMI) enables peoples with disabilities communicate the outer world by using electroencephalogram (EEG) signal. As a result, the visual stimulation is being recognized to be a most efficient stimulus for a multiple BMI system. Recently, steady-state visual evoked potential (SSVEP) is employed as a stimulator due to it plays an important role in the response to various visual parameters, i.e., flickering rate (F), intensity (I), and duty cycle (D). The SSVEPs are practical and useful in research because of its excellent signal-to- noise ratio and relative immunity to artifacts. However, another one of the possible paradigms, namely, the event-related potential (ERP) based BCI.  In this project, therefore, we offline investigated the effectiveness of SSVEP and ERP visual flickering paradigms for BCI system in terms of optimizing the parameter of visual simulation. Different experimental conditions were compared to achieve the best performance. Subjects were instructed to fixate LED light source then record associated SSVEP waveform on O1 and O2 channels. The variation in displaying of the presentation stimuli during a task was examined thereby demonstrating the high usability, adaptability and flexibility of the visual stimulator and determine the optimal visual parameters for the subject comfort.

\end{abstract}