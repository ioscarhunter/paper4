\chapter{Conclusion}

\label{ch:conclusions}

\section{Summary of Thesis Achievements}

\hspace{1.5cm}
In this project, we have developed the new alternative way to help the disabled people who can not move their part of the body but their brain still is functional by using EEG brain wave to interact with the equipment. We have studied in various techniques to use with ERPs and SSVEP to help the subject interact with the equipment. We have studied about the duty cycle,light intensity,light color and light luminance to use in our project.\par
We also have developed the hardware which is the visual stimulator combine with the tablet to command the program that we create for the subjects that we have already described in previous chapters. In ERPs experiment, we investigate the EEG data from subjects and use these data to calculate the accuracy. The conclusion of ERPs is when we obtain more trials of subjects , the accuracy rate will increase. In SSVEP experiment, to identify the flicker of visual stimuli we use windows function to separate the EEG data, use Fourier transform and subtract with baseline then use peak detection to determine the flicker and calculate the accuracy of the experiment.

\newpage
\section{Problems and obstacles}
\begin{itemize}
\item Hardware : We use the hardware that we design and create the hardware that uses in our experiment. However, we do not know about how to create this hardware and we use a lot of time to create it. The distance of each flicker can decrease the accuracy rate.

\item Environment : In our experiment, we have the chamber to investigate the EEG data of subjects but there is some light through in the chamber that maybe make some mistakes of data. The accuracy rate can be decreased when we use the experiment outside the chamber.

\item Algorithm : In this project, we use windows function and FFT algorithm to detect the frequency but it is not too much well-performed to detect the frequency.In our experiment, the lowest accuracy rate is 66.25\%. The further solution we plan to change the algorithm in our experiment to acquire good accuracy rate.
\end{itemize}

\section{Discussion}
\hspace{1.5cm} From our ERPs experiment,the best accuracy rate is 85\% and use 36 seconds to command one function of the program. We suggested this experiment give the high accuracy but it is time-consuming. We suggest the subject should use the previous experiment which uses only 28 seconds and 72.5\% accuracy. This can be the general case to use in this ERPs experiment.\\
\hspace{1.5cm} In SSVEP experiment, we have 2 experiment of SSVEP. The first experiment, we find which frequency is the best accuracy. We found the best accuracy of frequencies is between 13Hz to  18Hz. The 15Hz frequency is the best accuracy which is 83\%. The lowest accuracy which chosen is 69\% at 18Hz. We suggested that the lowest accuracy is not a bad rate  to use in next experiment. The second experiment, we use the result of the previous experiment in visual stimulator device. The best accuracy is 80\% at 13 Hz. We suggested that each visual stimuli flickers maybe close with others. The light of each flicker maybe interfere the light of flicker that subject is fixating. We can increase accuracy rate of this experiment, we can enlarge our device to reduce the interference. The accuracy of ERPs is better than SSVEP in our experiment. However, If we use the visual stimuli flickers in the same number for SSVEP and ERPs experiment, The ERPs experiment will take more time than the SSVEP experiment. We can choose to use ERPs or SSVEP in the proper experiment.

