\chapter{Problem description and related work}

\label{ch:Problem description and related work}

\setlength{\parindent}{4em}
\setlength{\parskip}{1em}
\renewcommand{\baselinestretch}{1.5}

\section{Problem description}
\hspace{1.5cm} Nowadays, there are many people who have been born with disabled body from various reasons. Most of them always have a limitation of their living that be different from ordinary people. At present, one of the main problems of disabled person is lack of help and ignore from other people. Using knowledge from scientific and engineering field to make a brain computer interface machine is the way to help these people to have the living like the ordinary people. For our project, we decide to assist the patient or physical disabled people that can't completely move their body part to interaction without helping from other people such as Paralysis, Spinal cord injured, Stoke and ALS.

\section{Review of related work}
2.2.1 Multitasking with BCI machine[1]
\\The multitasking with BCI machine experiment from EPFL Professor José del R. Millan. This experiment develops the BCI machine which allows the user to control the robot direction to move by using EEG signal and real-time communicate with other people via the camera that mount on the robot.\\
2.2.2 Real-time control of a robotic arm using a brain-computer interface[2]
\\The SSVEP control of robotic arm experiment from the Applied Signal Processing in Engineering and Neuroscience Lab (ASPEN Lab) of Old Dominion University.
This Lab is used to develop an application which using EEG headset to control the robotic arm. The subject should gaze on the one of the flickers in front of him for control or command the direction of robotic arm.\\
2.2.3	Visual stimuli for the P300 brain-computer interface: A comparison of white/grey and green/blue flicker matrices [3]
\\The P300 speller has mainly used white/grey flicker matrices as visual stimuli but they are not reducing the fatigue condition of user. Parra and colleagues evaluated what colour is reducing the fatigue condition of user. In their study, five single-color stimuli have been implemented. There are white, blue, red, yellow and green.
In this experiment, the green/blue chromatic flicker emerged as the safest and evoked the lowest rate of EEG spikes. The result showed that the accuracy rate was higher in response to the luminance chromatic flicker condition(LC) than in response to the luminance(L) or chromatic(C) flicker condition.
\\In conclusion, most users preferred green and blue make them feel less strain to the eyes. The subjects found that they could use green stimulus for longer periods as compared with red and blue in all frequency range.\\
2.2.4	Validation of the Emotiv EPOC® EEG gaming system for measuring research quality auditory ERPs [4]\\
There are two kinds of equipment for obtaining a brain signal as follow: Researching headset and Gaming headset. Studying this research can be concluded that the Gaming headset can be use as well as the Research headset because of the result of an experiment showed that the characteristics of the Gaming headset's graph are very similar to the graph of Research headset.\\
2.2.5 c-VEP Brain-Computer Interface (BCI) [5]
Code-modulated visual evoked potentials (c-VEP) is one of the kinds of BCI. c-VEP uses pseudorandom code to modulate different visual stimuli. The figure 2-4 A shows the configuration and modulation of the c-VEP BCI system\\
2.2.6	Human EEG responses to 1-100 Hz flicker [6]
Human EEG responses to 1-100 Hz flicker is stimulated in visual cortex. Herrmann reported that the SSVEP responses exhibited resonance phenomena around 10,20,40 and 80 Hz\\
2.2.7 SSVEP-based brain computer interface using the Emotiv EPOC//
In this research project, After the data was obtained from the headset, First the two occipital channels were averaged together. The averaging helped eliminate some of the noise between the two. Since SSVEP is based on a particular frequency being present in the EEG data, the signal processing should accomplish this by using a Fourier transform to convert from the time signals into the frequency domain after averaged the data. The result from converting is a frequency spectrum. Then, squaring the amplitude of each frequency component in this spectrum. This result was collected as active signal spectrum                      After that using a baseline removal method to further improve the results of signal isolation for the identification of SSVEP response. A baseline was recorded while the user was viewing a solid 50 percent gray screen without present any visual stimulation. After the data was recorded, Begin the method by using a Fourier transform with baseline spectrum and then reduce the noise by using smoothing filter. Finally, the smoothed baseline spectrum was subtracted from the active signal spectrum. Finally, the result from this is the spectrum that can classify the observed data.\\ 
2.2.8 A Survey of Stimulation Methods Used in SSVEP-Based BCIs
In this review paper, the stimulus frequency can be divided into three frequency bands. that is, low (1–12 Hz), medium (12–30 Hz) and high (30–60 Hz).The largest SSVEP amplitudes were observed near 10 Hz followed by 16–18 Hz so most of the SSVEP-based BCIs used the low and medium frequency bands.However, These two frequency band have disadvantages.First, subjective evaluations showed that frequencies between 5 and 25 Hz are more annoying than higher ones; visual fatigue would easily occur. Second, flash and pattern reversal stimuli can
provoke epileptic seizures especially in the 15–25 Hz range.Third, the low frequency band covers the alpha band (8–13 Hz) which can cause a considerable amount of false positives. All of these disadvantages can be avoided by using
the high frequency band    







