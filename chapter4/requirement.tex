\chapter{Requirement analysis and Software design}

\label{ch:Requirement analysis and Software design}

\setlength{\parindent}{4em}
\setlength{\parskip}{1em}
\renewcommand{\baselinestretch}{1.5}

\section{Requirement for application}

\subsection{Functional requirement}
\begin{itemize}
	\item The system allows user to control machine using brainwave.
    \item The system allows user to check the EEG headset status.
    \item The system allows administrator to see the authentication result.
    \item The system allows administrator to see real-time graph of each band.
    \item The system allows administrator to check the EEG headset status.
    \item The system allows user to save information into the system.
    \item The system shows the analysis result.
\end{itemize}  

\subsection{Non-Functional requirement}
\begin{itemize}
	\item The system uses C\# language to develop software.
    \item The system uses Windows Presentation form and Visual studio 2015 to create user interface.
    \item User friendly: The UI is look clear, simple, easy to use and understand.
    \item Performance: The system shows and records the EEG brainwave in real-time.
\end{itemize}

\section{Use case Diagram}

\section{Activity Diagram}
\subsection{}
