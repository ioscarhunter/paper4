\chapter{Conclusion}

\label{ch:conclusions}

\section{Summary of Thesis Achievements}

\hspace{1.5cm}
In this project, we have developed the new alternative way to help the disabled people who can not move their part of the body but their brain still is functional by using EEG brain wave to interact with the equipment. We have studied in various techniques to use with ERP and SSVEP to help the subject interact with the equipment. We have studied about the duty cycle,light intensity,light color and light luminance to use in our project.\par
We also have developed the hardware which is the visual stimulator combine with the tablet to command the program that we create for the subjects that we have already described in previous chapters. In ERP experiment, we investigate the EEG data from subjects and use these data to calculate the accuracy. The conclusion of ERP is when we obtain more trials of subjects , the accuracy rate will increase. In SSVEP experiment, to identify the flicker of visual stimuli we use windows function to separate the EEG data, use Fourier transform and subtract with baseline then use peak detection to determine the flicker and calculate the accuracy of the experiment.

\newpage
\section{Problems and obstacles}
\begin{itemize}
\item Hardware : We use the hardware that we design and create the hardware that uses in our experiment. However, we do not know about how to create this hardware and we use a lot of time to create it. The distance of each flicker can decrease the accuracy rate.

\item Environment : In our experiment, we have the chamber to investigate the EEG data of subjects but there is some light through in the chamber that maybe make some mistakes of data. The accuracy rate can be decreased when we use the experiment outside the chamber.

\item Algorithm : In this project, we use windows function and FFT algorithm to detect the frequency but it is not too much well-performed to detect the frequency.In our experiment, the lowest accuracy rate is 66.25\%. 
\end{itemize}